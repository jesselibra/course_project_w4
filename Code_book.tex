% Options for packages loaded elsewhere
\PassOptionsToPackage{unicode}{hyperref}
\PassOptionsToPackage{hyphens}{url}
%
\documentclass[
]{article}
\usepackage{lmodern}
\usepackage{amssymb,amsmath}
\usepackage{ifxetex,ifluatex}
\ifnum 0\ifxetex 1\fi\ifluatex 1\fi=0 % if pdftex
  \usepackage[T1]{fontenc}
  \usepackage[utf8]{inputenc}
  \usepackage{textcomp} % provide euro and other symbols
\else % if luatex or xetex
  \usepackage{unicode-math}
  \defaultfontfeatures{Scale=MatchLowercase}
  \defaultfontfeatures[\rmfamily]{Ligatures=TeX,Scale=1}
\fi
% Use upquote if available, for straight quotes in verbatim environments
\IfFileExists{upquote.sty}{\usepackage{upquote}}{}
\IfFileExists{microtype.sty}{% use microtype if available
  \usepackage[]{microtype}
  \UseMicrotypeSet[protrusion]{basicmath} % disable protrusion for tt fonts
}{}
\makeatletter
\@ifundefined{KOMAClassName}{% if non-KOMA class
  \IfFileExists{parskip.sty}{%
    \usepackage{parskip}
  }{% else
    \setlength{\parindent}{0pt}
    \setlength{\parskip}{6pt plus 2pt minus 1pt}}
}{% if KOMA class
  \KOMAoptions{parskip=half}}
\makeatother
\usepackage{xcolor}
\IfFileExists{xurl.sty}{\usepackage{xurl}}{} % add URL line breaks if available
\IfFileExists{bookmark.sty}{\usepackage{bookmark}}{\usepackage{hyperref}}
\hypersetup{
  pdftitle={Code\_Book},
  pdfauthor={Jesse Libra},
  hidelinks,
  pdfcreator={LaTeX via pandoc}}
\urlstyle{same} % disable monospaced font for URLs
\usepackage[margin=1in]{geometry}
\usepackage{graphicx,grffile}
\makeatletter
\def\maxwidth{\ifdim\Gin@nat@width>\linewidth\linewidth\else\Gin@nat@width\fi}
\def\maxheight{\ifdim\Gin@nat@height>\textheight\textheight\else\Gin@nat@height\fi}
\makeatother
% Scale images if necessary, so that they will not overflow the page
% margins by default, and it is still possible to overwrite the defaults
% using explicit options in \includegraphics[width, height, ...]{}
\setkeys{Gin}{width=\maxwidth,height=\maxheight,keepaspectratio}
% Set default figure placement to htbp
\makeatletter
\def\fps@figure{htbp}
\makeatother
\setlength{\emergencystretch}{3em} % prevent overfull lines
\providecommand{\tightlist}{%
  \setlength{\itemsep}{0pt}\setlength{\parskip}{0pt}}
\setcounter{secnumdepth}{-\maxdimen} % remove section numbering

\title{Code\_Book}
\author{Jesse Libra}
\date{6/6/2020}

\begin{document}
\maketitle

\hypertarget{code-book}{%
\subsection{Code Book}\label{code-book}}

\#\#\#Data The data are from the Human Activity Recognition Using
Smartphones Dataset:
\url{http://archive.ics.uci.edu/ml/datasets/Human+Activity+Recognition+Using+Smartphones}

Download here:
\url{https://d396qusza40orc.cloudfront.net/getdata\%2Fprojectfiles\%2FUCI\%20HAR\%20Dataset.zip}

\hypertarget{variables-and-data-created}{%
\subsubsection{Variables and data
created}\label{variables-and-data-created}}

``x\_test'', ``x\_train'', ``y\_test'', ``y\_train'', ``features'',
``activityLbs'': original data from the zip

``dataset'': test and train data combined.

``datasetms'': dataset with just mean and standard deviation columns.\\
``join'': Join adding activity labels to the datasetms\\
``datasummary'': means by subject, activity, and measure

\#\#\#Description of run\_analysis.R 1. Merges the training and the test
sets to create one data set. 2. Extracts only the measurements on the
mean and standard deviation for each measurement. 3. Uses descriptive
activity names to name the activities in the data set 4. Appropriately
labels the data set with descriptive variable names (in step 1). 5. From
the data set in step 4, creates a second, independent tidy data set with
the average of each variable for each activity and each subject.

\end{document}
